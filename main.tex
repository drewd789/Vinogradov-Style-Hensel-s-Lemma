\documentclass{article}
\usepackage[utf8]{inputenc}

\usepackage{amsmath}
\usepackage{amsfonts}
\usepackage{mathtools}
\usepackage{amsthm}
\usepackage{amssymb}

\newtheorem{theorem}{Theorem}
\newtheorem{lemma}{Lemma}
\newtheorem{corollary}{Corollary}

\DeclarePairedDelimiter\floor{\lfloor}{\rfloor}

\title{Vinogradov Style Hensel's Lemma for Additive Forms}
\author{Drew Duncan}
\date{\today}

\begin{document}

\maketitle

\section{Introduction}
This is a relatively elementary proof of Hensel's Lemma for additive forms which I took from Vinogradov \cite{vinogradov2016elements}, chapter 2, lemma 8.  (I first found reference to Vinogradov's proof in \cite{davenport1963homogeneous}.)  Vinogradov's own exposition of this proof leaves out some details, which I attempt to fill in below.

In this form, Hensel's Lemma gives conditions when a solution to an additive form over $\mathbb{Q}_p$ modulo some power of $p$ can be raised to a solution in $\mathbb{Q}_p$.  This proof can be extended in a straightforward way to extensions of $\mathbb{Q}_p$ given some data on the extension (in this case, raising a solution modulo some power of the uniformizer $\pi$). 

However, there are some surprising details which arise in this extension of the proof.  The first is that the proof is substantially different from that given in \cite{leep2018diagonal}, which is the only proof known to me in the literature.  Second, the power depends on different information about the extension than in the other proof.  Finally, and most surprisingly, in some cases the power of $\pi$ needed to raise a solution is lower than that obtained in the other proof.

\section{Additive Forms over $\mathbb{Q}_p$}

\begin{theorem}
Let $p>2$ and  $\gamma = \tau+1$ for $d=m p^\tau$ with $p \nmid m$ if $\tau \ge 1$, $\gamma = 1$ otherwise.  Suppose $a_1 x_1^d + \ldots + a_s x_s^d \equiv a \pmod{p^\gamma}$ with at least one $x_i$ not divisible by $p$.  Then there exists $y_1, \ldots, y_s$ such that $a_1 y_1^d + \ldots a_s y_s^d \equiv a \pmod{p^\delta}$ for any $\delta \ge \gamma$.
\end{theorem}

\begin{proof}
By a suitable change of variables, it suffices to show that $x^d \equiv a \pmod{p^\gamma}$ implies that there exists $y$ such that $y^d \equiv a \pmod{p^\delta}$ for and $\delta \ge \gamma$.  Suppose this holds.  Let $g$ be a generator for the cyclic group $(\mathbb{Z}/p^\delta\mathbb{Z})^\times = \langle g\rangle$.  Then we have $x^d g^k \equiv a \pmod{p^\delta}$ for some exponent $k$.

First, note that $g^k \equiv 1 \pmod{p^\gamma}$.  Considering $g$ now to be the element obtained reducing modulo $p^\gamma$,  then $g$ is a generator of $(\mathbb{Z}/p^\delta\mathbb{Z})^\times$.  It follows that $(p-1)p^{(\gamma-1)} \mid k$, or more precisely, $p^\tau \mid k$.  We now need only show that $m \mid k$.

Next, note that because the order of $(\mathbb{Z}/p^\delta\mathbb{Z})^\times$ is $(p-1)p^{(\delta-1)}$, we can replace $k$ with $k' = k + i(p-1)p^{(\delta-1)}$ for any $i$ and all of the above congruences continue to hold.  Factoring out $(p-1)$ from both terms, we get $k' = (p-1)(b + ip^c)$ for some $b$ and $c$.  Solving for $i$ in $b + ip^c \equiv 0 \pmod{m}$ gives $i$ such that $d \mid k'$.
\end{proof}

Notice that this proof depended on knowing the order of a generator of the group of units modulo some power of $p$.  If $p^{\tau}$ divides the order of the generator of $(\mathbb{Z}/p^\delta\mathbb{Z})^\times$, then the result readily follows.  The minimal power of $p$ for which this happens is easy to obtain because these groups are cyclic.

For $p=2$, the proof is a bit more subtle.  Although $(\mathbb{Z}/2\mathbb{Z})^\times$ and $(\mathbb{Z}/4\mathbb{Z})^\times$ are cyclic, from then on the groups of units are not generated by a single element.  However, this is not difficult to overcome.

\begin{theorem}
Let $\gamma = \tau + 2$ for $d=m 2^\tau$ with $2 \nmid m$ if $\tau \ge 1$, $\gamma = 1$ otherwise.  Suppose $a_1 x_1^d + \ldots + a_s x_s^d \equiv a \pmod{p^\gamma}$.  Then there exists $y_1, \ldots, y_s$ such that $a_1 y_1^d + \ldots a_s y_s^d \equiv a \pmod{2^\delta}$ for any $\delta \ge \gamma$.
\end{theorem}

\begin{proof}
For any group of units modulo a power of 2, the elements congruent to 1 modulo 4 are generated by 5.  Thus, any element of $\alpha \in (\mathbb{Z}/2^\delta\mathbb{Z})^\times$ can be written $\alpha = 5^k 3^j$ with $0 \le k < 2^{(\delta-2)}$ and $j \in \{0,1\}$.

The case $\tau = 0$ is trivial.  If $\tau \ge 1$, then $\gamma \ge 3$, and so $2 \mid j$ which implies that $j=0$.  The argument for $d \mid k$ follows exactly as in the $p > 2$ case.
\end{proof}

Now we move on to extensions of $\mathbb{Q}_p$.  We state a result without proof.

\begin{theorem}\label{leep}
Let $K$ be a finite extension of $\mathbb{Q}_p$ with degree of ramification $e$, $\pi$ be a uniformizer for $K$, and $d=m p^\tau$ and $\gamma = \floor*{\frac{e}{p-1}} + e\tau + 1$.  A solution modulo $\pi^\gamma$ lifts to a solution in $K$.
\end{theorem}

The only proof I have is due to Dr.\ Leep and Sordo-Viera, and involves a lot of algebraic manipulations that don't give me much insight into the result.  If one can compute generators and their orders for the group of units modulo a power of $\pi$, then one can proceed with the ``Vinogradov-style" arguments above, finding a power of $\pi$ so that the orders are divisible by $p^\tau$, or have an order that is eventually finite.  I will attempt a simple argument for totally ramified extensions of $\mathbb{Q}_2$.  The surprise is that the $\gamma$ given in Theorem \ref{leep} is not optimal.

Let's look at some examples for ramified quadratic extensions of $\mathbb{Q}_2$.

Okay, now let's look at the generators and their orders modulo powers of $\pi$.

You'll notice a pattern in the orders.

\bibliographystyle{unsrt}
\bibliography{biblio}

\end{document}
